\newtheorem{problem}{Problem}[section]
%\newcommand{\inch}{\ensuremath{\,\textrm{in.}}}

\chapter{One-dimensional Steady State Problems}

\section{Overview}
This Chapter mainly discuss different kind of one-dimensional steady state problems.

\section{Fourier's law}

\begin{example}
\textbf{Fourier’s law, plane wall, constant internal heat
generation.} \textcolor{blue} {\emph{Refer to tutorial Fourier’s
law.nb, and Ch. 2, P. 69.}}

Consider a one-dimensional plane wall in steady state and without heat generation.
Temperature on one side is $T_0=50~K$, other side is $T_1=30~K$, wall width is $L=0.01~m$,
area is $A=1~m^2$, thermal conductivity $k=0.5~W/m.k$. Sketch the heat distribution on T-x coordinate,
what is the heat flux  through the wall?
\end{example}

\begin{solution}
For the plan wall, the heat resistance is
$$R=\frac{L}{kA}$$
So the heat rate 
$$
Q=\frac{\delta T}{R}=kA\frac{T_0-T_1}{L}=1000~W
$$

\begin{lstlisting}
Plot1=Plot[$T_0$ + x/L*($T_1-T_0$),{x, 0, L}]
\end{lstlisting}
The heat flux$$q=\frac{Q}{A}==1000~W/m^2$$
\begin{lstlisting}
R=L/(k*A); (*Resistance*)
$T_0$=50; (*Temperature on the other side of the wall)
$T_1$=30;
Q=($T_0$-$T_1$)/R; (*Watt5s*)
FLUX=Q/A;
\end{lstlisting}
Based on Fourier’s Law, the heat rate
\begin{eqnarray*}
Q=-kA\frac{dT}{dx}\\
dT=-\frac{Q}{ka}dx
\end{eqnarray*}
~\\
Integral
$$\int dT=-\int \frac{Q}{ka} dx$$
In steady state,  is constant, so the heat distribution along the x direction is
$$T(x)=C_1x+C_2$$
As given, $T(0)=T_0=50~K, T(0.01)=T_1=30~K$, we can get
$$T(x)=\frac{T_1-T_0}{L}x+T_0=-2000x+50$$
And the sketch of  is shown below:
\begin{lstlisting}
Plot2=ListPlot[{{0,50},{0.01,30}},
      PlotStyle->{AbsolutePointSize[8]},
      AxesLabel->{Distance (m),Temperature}]
\end{lstlisting}
\end{solution}

\begin{example}
\textbf{Conical section} \textcolor{blue} {\emph{Refer to tutorial 
Prob\_ 7 copy.nb, and Ch. 3, P. 134.}}

A diagram shows a conical section fabricated from pyroceram, $k=3.46W/m.K$.
It is of circular cross section with the diameter $D=ax$, where $a=0.25$.
The small end is at $x_1=50~mm$ and the large end at $x_2=250~mm$.
The end temperature are $T_1=400~K$ and $T_2=600~K$, while the lateral 
surface is well insulated.
\begin{enumerate}
\item Derive and expression for the temperature distribution $T(x)$,
assuming one-dimensional conditions. Sketch the temperature distribution.
\item Calculate the heat rate $Q$ through the cone.
\item If $a$ changes from 0.001 to 1, sketch change of $Q$.
\end{enumerate}
\end{example}

\begin{solution}
\begin{enumerate}
\item
Consider the heat conduction is under steady state, one-dimensional coordinate,
without internal heat generation, the heat transfer rate is a constant independent
of $x$. Use Fourier’s Law to determine the temperature distribution.
$$Q=-kA\frac{dT}{dx}$$
Where $A=\frac{\pi D^2}{4}=\frac{\pi a^2x^2}{4}$. Separating variables,
$$\frac{4Qdx}{\pi a^2x^2}=-kdT$$
Integrating from $x_1$ to any $x$ within the cone, and recalling that $Q$ and $k$
are constant, it follows that
$$\frac{4Q}{\pi a^2}\int_x^{x_1} dx/x^2=-k\int dT$$
Hence
$$\frac{4Q}{\pi a^2}\left(-\frac{1}{x}+\frac{1}{x_1}\right)=-k\left(T-T_1\right)$$
and
$$T(x)=T_1-\frac{4Q}{\pi a^2}\left(\frac{1}{x_1}-\frac{1}{x}\right)$$
Although $Q$ is a constant, it is as yet an unknown.
However, it may be determined by evaluating the above expression at
$x=x_2$, where $T(x_2)=T_2$. Hence
$$Q=\frac{\pi a^2k(T_1-T_2)}{4[(1/x_1)-(1/x_2)]}$$
and solving for Q
\begin{lstlisting}
Q =*a^2*k*(- )/(4((1/x1)-(1/x2))) (*  Watts  *)
\end{lstlisting}
Substituting for $Q$ into the expression for $T(x)$,
the temperature distribution becomes
$$T_2=T_1+(T_1-T_2)\left[\frac{(1/x)-(1/x_2)}{(1/x_1)-(1/x_2)}\right]$$
\begin{lstlisting}
Q =*a^2*k*(- )/(4((1/x1)-(1/x2))) (*  Watts  *)
\end{lstlisting}
From the result, temperature may be calculated as a function of $x$
and the distribution is as shown.
\begin{lstlisting}
Plot2 = Plot[T, {x, .05, .25}, PlotRange -> All, 
AxesLabel -> {Distance (m), Temperature }]
\end{lstlisting}
\item
Substituting numerical values into the foregoing result for the heat 
transfer rate, it follows that
$$Q=\frac{\pi 0.25^2\times3.46~\text{W/m.K}\times(400-600)~K}{4(1/0.05~m - 1/0.25~m)}=-2.12W$$
\item
If changes from $0.001$ to $1$, as $Q$ has expression changes with $a$,
we can sketch $Q$’s changes with $a$.
\begin{lstlisting}
Qax =  *x^2*   k*(- )/(4 ((1/x1) - (1/x2)))
PlotQ = Plot[Qax, {x, .001, 1}, PlotRange -> All,    AxesLabel -> {"a", "Q" }]
\end{lstlisting}
\end{enumerate}
\end{solution}

\section{1-D steady state, constant	internal heat generation problems}
\begin{example}
\textcolor{blue} {\emph{Refer to tutorial HW\_1\_MMATICA.nb}}
A large thin slab of thickness $L=0.1~m$ is “setting.” Setting is an exothermic process 
that releases $\dot{q}=100W/m^3$. Here the slab heat conductivity is in steady state. 
Set the x-axis along with the wall thickness. At position $x=0m$, temperature is $T_0=37~K$,
at position $x=L$,$T_1=33~K$, thermal conductivity $k=0.4W/m.K$. What’s the
temperature distribution in along the length of the slab? 
\end{example}

\begin{solution}
Based on the heat diffusion equation 
$$\nabla^2 T+\frac{\dot{q}}{k}=\frac{1}{\propto}\frac{\partial T}{\partial t}$$
where
$$\nabla^2 T \equiv \frac{\partial^2 T}{\partial x^2}+
\frac{\partial^2 T}{\partial y^2}+
\frac{\partial^2 T}{\partial z^2}$$
In this problem, the large thin slab could be considered as a one-dimensional problem 
with only $x$ dimension, and consider the slab is in steady state with no change with $t$,
so the one-dimensional heat diffusion equation for the slab could be write as
$$\frac{d^2 T}{d x^2}=-\frac{\dot{q}}{k}$$
\begin{lstlisting}
solution1 = NDSolve[{T''[x] + /k == 0, T[0] == ,  T[0.1] == }, T, {x, 0, 0.1}];
\end{lstlisting}
Integration twice
$$T(x)=-\frac{\dot{q}}{2k}x^2+C_1x+C_2$$
By evaluating $T(0)=T_0$, and $T(L)=T_1$, hence
$$T(x)=-\frac{\dot{q}}{2k}x^2+ \left(\frac{T_1-T_0}{L}+\frac{\dot{q}L}{2k}\right)x+T_0$$
From the result, temperature distribution could be expressed as a quadratic curve as below
\begin{lstlisting}
PS1 = Plot[Evaluate[T[x] /. First[solution1]], {x, 0, 0.1},    PlotRange -> All]
\end{lstlisting}
\end{solution}

\begin{example}
\textcolor{blue} {\emph{Refer to tutorial HW\_1\_MMATICA.nb, and Ch. 3, P. 134.}}
~\\
A steady state long tube generating thermal energy at a uniform volumetric rate
$\dot{q}=1000W/m^3$, the thermal conductivity $k=0.4$W/m.K.
At radius $r_1=0.1368m$, temperature $T_0=37~K$, at position $r_2=0.1768m$
, temperature $T_1=33~K$, the two end of the rod are well insulated.
What is the temperature distribution along the radius of the tube?
\end{example}

\begin{solution}
The heat diffusion equation for cylindrical system is 
$$ 
\frac{1}{r}\frac{\partial}{r}(kr\frac{\partial T}{\partial r})+
\frac{1}{r^2}\frac{\partial}{\phi}(k\frac{\partial T}{\partial \phi})+
\frac{\partial}{z}(k\frac{\partial T}{\partial z})+
\dot{q} =
\rho c_p\frac{\partial T}{\partial t}
$$
In this problem, consider the heat distribution change only on $r$ direction,
and since the tube is in steady state, the temperature distribution would not
change with time $t$. The heat distribution equation could be written as
$$\frac{1}{r}\frac{\partial}{r}(kr\frac{\partial T}{\partial r})+\dot{q}=0$$
Integration twice
$$T(r)=-\frac{\dot{q}}{4k}r^2+C_1Inr+C_2$$
Substituting $T(r_1)=T_0$, $T(r_2)=T_1$, hence
$$
T(r)=-\frac{\dot{q}}{4k}r^2 +\left[\frac{T_1-T_0+(\dot{q}/4k)\cdot r_2^2}{In(r_2/r_1)} \right]\cdot Inr +
\frac{T_0Inr_2-T_1Inr_1-(\dot{q}/4k)\cdot r_2^2}{In(r_2/r_1)}
$$
~\\
By evaluating $r_1=0.1368~m, T_0=37~K, r_2=0.1768~m \text{and} T_1=33~K$,
sketch the $T(r)$
\end{solution}

\section{1-D steady state, variate internal	heat generation	problems}

\section{Fin problems}
\begin{example}
A rectangular fin with uniform cross-sectional area has constant heat conductivity 
$k=3~W/m.K$ heat conduction coefficient $h=10~W/mc$, width is $W=0.01~m$, 
Thickness is $L=1~m$. At start position $x_1=0~m$, temperature is $T_1=30^\circ C$.
The surface of the fin is exposed to ambient air at $10~K$ with a convection heat transfer
coefficient $h=10~W/m^2K$. Plot the temperature distribution in the fin under below conditions.
\begin{enumerate}
\item Prescribed tip temperature: at the end position $x_2=0.04m, T_2=40^\circ C$
\item Adiabatic tip condition.
\item Infinite tip condition.
\end{enumerate}

\end{example}

\begin{solution}
~\\
\begin{enumerate}
\item
The fin energy balance equation
$$$$
For the proscribed fin problem with constant cross section area , and surface area, we have  ,  . Hence 

To simplify the form of this eqauation, we transform the dependent variable by defining an excess temperature  as 

Where since  is constant, . Then we obtain

Where 

With prescribed boundrary condition

Then for prescribed condition the fin temperature distribution is
\item

For adiabatic tip condition, the heat convection rate at the tip is considered negligible,

And 

Then the heat distribution equation of adiabatic tip condition could be written as

The temperature distribution along x direction is shown below
\item
For an infinite fin the tip is , and the boundary condition at the tip is 

And the temperature distribution is 

Sketch the temperature distribution 
\end{enumerate}

\end{solution}

\begin{example}
\textbf{Constant k and A}
A rectangular fin with uniform cross-sectional area has constant heat conductivity heat conduction coefficient , width is , Thickness is . At start position , temperature is . The surface of the fin is exposed to ambient air at  with a convection heat transfer coefficient . Plot the temperature distribution in the fin under below conditions.
\end{example}

\begin{solution}
TODO
\end{solution}


